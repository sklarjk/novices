\section{Creating a PTX doc}
Put here the instructions for creating \textit{hello-world.ptx}.

\begin{itemize}
\item
Note that Sublime or the like should be used to create/edit PTX docs.
\item Note that you will need to use linux commands to make sure you are calling xsltproc and pointing to the style and PTX files in the correct locations.
\end{itemize}


\section{Inclusion}\label{inc}

If you are creating a long document, you may want to write parts of it in modular PTX files and include those in your master document. To include a file \textit{chap1.ptx} in file \textit{master.ptx}, put the command
\texttt{$<$xi:include href="./chap1.pts" /$>$}
in the body of \textit{master.ptx} where you want the included file's material to appear. The \textit{chap1.ptx} document should only contain exactly the content you want to have appear in \textit{master.ptx}.

\section{knowls and their switches}
Explain what knowls are and that you  hide/unhide them by modifying the code in \textit{mathbook-html.xsl}  and/or\textit{ mathbook-latex.xsl}.
